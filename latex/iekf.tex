\documentclass[12pt, a4paper]{article}
\usepackage{graphicx} % Required for inserting images
\usepackage{bm}
\usepackage{amsmath}
\usepackage{amssymb}
\usepackage[top=2cm, bottom=2cm, left=2cm, right=2cm]{geometry}
\usepackage{indentfirst}
\usepackage{mathtools}

\DeclareMathOperator{\Exp}{Exp}
\DeclareMathOperator{\Log}{Log}

% Define \mathpzc font lower case, \mathcal is for upper case
\DeclareFontFamily{OT1}{pzc}{}
\DeclareFontShape{OT1}{pzc}{m}{it}{<-> s * [1.10] pzcmi7t}{}
\DeclareMathAlphabet{\mathpzc}{OT1}{pzc}{m}{it}


\title{Invariant Extended Kalman Filter for SE3 Localization}
\author{Yi-Chen Zhang}
\date{\today}

\begin{document}

\maketitle

\section{Introduction}
We consider the vehicle moving on the 2D plane in the 3D space. The robot
receives control actions in the form of axial and angular velocities, and is
able to measure its position using a GPS for instance.

\section{State representation and motion control}
The robot pose $\mathpzc{x}$ is in SE(3),
\[
  \mathpzc{x}=\begin{bmatrix}
    \bm{R} & \bm{t}\\
    \bm{0}^{T} & 1
  \end{bmatrix}\in \text{SE(3)}
\]
where $\bm{R}\in$ SO(3) is the rotation and $\bm{t}\in\mathbb{R}^{3}$ is the
translation.

The control signal $\bm{u}$ is a twist in $\mathfrak{se}(3)$ comprising
linear velocity $\bm{\nu}$ and angular velocity $\bm{\omega}$, with no
lateral and vertical velocity components, integrated over the sampling time
$\Delta t$.
\[
  \bm{u} = \begin{bmatrix}
    \bm{\nu}\Delta t\\
    \bm{\omega}\Delta t
  \end{bmatrix}\in\mathfrak{se}(3)
\]
where $\bm{\nu}=[\nu_{x}, 0, 0]^{T}$ and $\bm{\omega}=[0, 0, \omega_{z}]^{T}$.

The control is corrupted by additive Gaussian noise $\bm{\epsilon}$, with
covariance $\bm{Q}$.
\[
  \bm{Q}=\begin{bmatrix}
    \sigma_{x}^{2} & 0 & 0 & 0 & 0 & 0\\
    0 & \sigma_{y}^{2} & 0 & 0 & 0 & 0\\
    0 & 0 & \sigma_{z}^{2} & 0 & 0 & 0\\
    0 & 0 & 0 & \sigma_{\theta}^{2} & 0 & 0\\
    0 & 0 & 0 & 0 & \sigma_{\gamma}^{2} & 0\\
    0 & 0 & 0 & 0 & 0 & \sigma_{\psi}^{2}
  \end{bmatrix}
\]
This noise account for possible slippage.

\section{State transition function and its derivative}
At the arrival of a control $\bm{u}$, the vehicle pose is updated with the
motion equation
\[
  \mathpzc{x}_{\mkern4mu pred} = \bm{f}(\mathpzc{x}, \bm{u}) = \mathpzc{x}\oplus\bm{u}=
  \mathpzc{x}\Exp(\bm{u})
\]
\[
  \bm{F}=
  \frac{\prescript{\mathpzc{x}}{}{D}\bm{f}(\mathpzc{x},\bm{u})}{D\mathpzc{x}}=
  \frac{\prescript{\mathpzc{x}}{}{D\mathpzc{x}\oplus\bm{u}}}{D\mathpzc{x}}=
  \bm{J}^{\mathpzc{x}\oplus\bm{u}}_{\mathpzc{x}} =
  \textbf{Ad}_{\Exp(\bm{u})}^{-1}
\]
\[
  \bm{W}=
  \frac{\prescript{\mathpzc{x}}{}{D}\bm{f}(\mathpzc{x},\bm{u})}{D\bm{u}}=
  \frac{\prescript{\mathpzc{x}}{}{D\mathpzc{x}\oplus\bm{u}}}{D\bm{u}}=
  \bm{J}^{\mathpzc{x}\oplus\bm{u}}_{\bm{u}}=\bm{J}_{r}(\bm{u})
\]


\section{Measurement function and its derivative}
The GPS measurement is in $\mathbb{R}^{3}$, denoted as $\bm{y}$, and is put in
Cartesian form for simplicity. The measurement noise $\bm{\delta}$ is zero mean
Gaussian, and is specified with a covariance matrix $\bm{R}$.

We notice the rigid motion action $\bm{y}=\bm{h}(\mathpzc{x})=\mathpzc{x}\cdot
\bm{0}+\bm{\delta}$, which is the vehicle coordinates in the global frame. The
measurement equation is
\[
  \bm{y} = \bm{h}(\mathpzc{x}) = \mathpzc{x}\cdot\bm{0}
\]

\begin{align*}
  \bm{H} & =
  \frac{\prescript{\scriptscriptstyle\mathcal{E}}{}{D}\bm{h}(\bm{\mathpzc{x}})}
  {D \mathpzc{x}}\\
  & = \lim_{\bm{\tau}\rightarrow \bm{0}} \frac{\bm{h}(\bm{\tau}\oplus\mathpzc{x})
  \ominus \bm{h}(\mathpzc{x})}{\bm{\tau}}\\
  & = \lim_{\bm{\tau}\rightarrow \bm{0}} \frac{(\Exp(\bm{\tau})\mathpzc{x})\cdot
  \bm{0}-\mathpzc{x}\cdot \bm{0}}{\bm{\tau}}\\
  & = \lim_{\bm{\tau}\rightarrow \bm{0}}\frac{\left(\begin{bmatrix}
        \bm{I}+\bm{\phi}^{\wedge} & \bm{\rho}\\
        \bm{0}^{T} & 1
  \end{bmatrix}\begin{bmatrix}
    \bm{R} & \bm{t}\\
    \bm{0}^{T} & 1
  \end{bmatrix}\right) \cdot \bm{0}-\begin{bmatrix}
    \bm{R} & \bm{t}\\
    \bm{0}^{T} & 1
  \end{bmatrix}\cdot \bm{0}}{\bm{\tau}}\\
  & = \lim_{\bm{\tau}\rightarrow \bm{0}}\frac{\begin{bmatrix}
      (\bm{I}+\bm{\phi}^{\wedge})\bm{R} &
      (\bm{I}+\bm{\phi}^{\wedge})\bm{t}+\bm{\rho}\\
      \bm{0}^{T} & 1
    \end{bmatrix}\cdot \bm{0}-\begin{bmatrix}
      \bm{R} & \bm{t}\\
      \bm{0}^{T} & 1
  \end{bmatrix}\cdot \bm{0}}{\bm{\tau}}\\
  & = \lim_{\bm{\tau}\rightarrow
  \bm{0}}\frac{(\bm{I}+\bm{\phi}^{\wedge})\bm{t}+\bm{\rho}-\bm{t}}{\bm{\tau}}\\
  & = \lim_{\bm{\tau}\rightarrow
  \bm{0}}\frac{\bm{\phi}^{\wedge}\bm{t}+\bm{\rho}}{\bm{\tau}}\\
  & = \frac{\partial(\bm{\phi}^{\wedge}\bm{t}+\bm{\rho})}{\partial
  \bm{\tau}}\Bigg|_{\bm{\tau}=\bm{0}}\\
  & = \left[\left.\frac{\partial(\bm{\phi}^{\wedge}\bm{t}+\bm{\rho})}
      {\partial\bm{\rho}}\right|_{\bm{\rho}=\bm{0}},
      \left.\frac{\partial(\bm{\phi}^{\wedge}\bm{t}+\bm{\rho})}
      {\partial\bm{\phi}}\right|_{\bm{\phi}=\bm{0}}\right]\\
  & = [\bm{I}, -\bm{t}^{\wedge}]
\end{align*}

\[
  \bm{V}=\frac{\prescript{\scriptscriptstyle\mathcal{E}}{}{D}\bm{\delta}}
  {D \bm{\delta}}=\bm{I}
\]

\section{Kalman filter algorithm}
Now we are ready to implement invariant extended Kalman filter (IEKF) for the
SE(3) localization. Given the initial state $\mathpzc{x}_{\mkern4mu 0}$ and
state covariance $\bm{P}_{0}$, the IEKF algorithm is summarized as follows:\\

\textbf{Prediction}:
\begin{center}
\begin{tabular}{ll}
  Predicted state estimate: & $\mathpzc{x}_{\mkern4mu pred, t}=
  \mathpzc{x}_{\mkern4mu t-1}\oplus\bm{u}_{t}$\\
  Predicted error covariance: & $\prescript{\mathpzc{x}}{}{\bm{P}}_{pred, t}=
  \bm{F}_{t}\prescript{\mathpzc{x}}{}{\bm{P}}_{t-1}\bm{F}_{t}^{T}+
  \bm{W}_{t}\bm{Q}_{t}\bm{W}_{t}^{T}$
\end{tabular}
\end{center}

\textbf{Update}:
\begin{center}
\begin{tabular}{ll}
  Right to left invariant: &
  $\prescript{\scriptscriptstyle\mathcal{E}}{}{\bm{P}}_{pred, t}=
  \textbf{Ad}_{\mathpzc{x}}\prescript{\mathpzc{x}}{}{\bm{P}}_{pred, t}
  \textbf{Ad}_{\mathpzc{x}}^{T}$\\
  Innovation: & $\bm{z}_{t}=\bm{y}_{t}-\bm{h}(\mathpzc{x}_{\mkern4mu pred,t})$\\
  Innovation covariance: & $\bm{S}_{t}=
  \bm{H}_{t}\prescript{\scriptscriptstyle\mathcal{E}}{}{\bm{P}}_{pred,t}
  \bm{H}_{t}^{T}+\bm{V}_{t}\bm{R}_{t}\bm{V}_{t}^{T}$\\
  Kalman gain: & $\bm{K}_{t}=
  \prescript{\scriptscriptstyle\mathcal{E}}{}{\bm{P}}_{pred,t}
  \bm{H}_{t}^{T}\bm{S}_{t}^{-1}$\\
  Observed error: & $\delta \bm{x}_{t}=\bm{K}_{t}\bm{z}_{t}$\\
  Updated state estimate: & $\mathpzc{x}_{\mkern4mu t} = \delta \bm{x}_{t}\oplus
  \mathpzc{x}_{\mkern4mu perd,t}$\\
  Updated error covariance: &
  $\prescript{\scriptscriptstyle\mathcal{E}}{}{\bm{P}}_{t}=
  (\bm{I}-\bm{K}_{t}\bm{H}_{t})
  \prescript{\scriptscriptstyle\mathcal{E}}{}{\bm{P}}_{pred,t}$\\
  Left to right invariant: &
  $\prescript{\mathpzc{x}}{}{\bm{P}}_{t}=
  \textbf{Ad}_{\mathpzc{x}^{-1}}
  \prescript{\scriptscriptstyle\mathcal{E}}{}{\bm{P}}_{t}
  \textbf{Ad}_{\mathpzc{x}^{-1}}^{T}$\\
\end{tabular}
\end{center}

For the updated error covariance, the Joseph formula should be employed for numerical stability. This can be expressed as follows:
\[
  \prescript{\scriptscriptstyle\mathcal{E}}{}{\bm{P}}_{t}=
  (\bm{I}-\bm{K}_{t}\bm{H}_{t})
  \prescript{\scriptscriptstyle\mathcal{E}}{}{\bm{P}}_{pred,t}
  (\bm{I}-\bm{K}_{t}\bm{H}_{t})^{T}+
  \bm{K}_{t}\bm{V}_{t}\bm{R}_{t}\bm{V}_{t}^{T}\bm{K}_{t}^{T}
\]

In the case where the measurement model is correct, the Kalman filter utilizes it for updates. Thus, a conditional statement for data association is introduced. The Mahalanobis distance is calculated for the measurement residual to determine if the measurement is suitable for updating:
\[
  (\bm{y}_{t}-\bm{h}(\mathpzc{x}_{\mkern4mu pred,t}))^{T}
  \bm{S}_{t}^{-1}(\bm{y}_{t}-\bm{h}(\mathpzc{x}_{\mkern4mu pred,t}))
  \leq D_{th},
\]
Here, $D_{th}$ represents a predetermined threshold. One can further show that the quadratic term is actually a $\chi_{m}^{2}$ distribution, where $m$ is the number of measurements.

\end{document}
